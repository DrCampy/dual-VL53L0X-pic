Description of the \iic registers.

\iffalse
 \subsection{Overview}
 \begin{multicols}{2}
  \centering{Configuration}
  \begin{itemize}
   \item \texttt{CONFIG\_L \quad 0x00}
   \item \texttt{CONFIG\_H \quad 0x01}
   \item \texttt{ADDRESS \quad 0x02}

  \end{itemize}

  \columnbreak
  \centering{Data}
  \begin{itemize}
   \item \texttt{RIGHT \quad 0x10}
   \item \texttt{LEFT \quad 0x11}
   \item \texttt{MIN \quad 0x12}
   \item \texttt{MAX \quad 0x13}
   \item \texttt{AVG \quad 0x14}
  \end{itemize}

 \end{multicols}
\fi
\subsection{Configuration Registers}
\subsubsection{\texttt{CONFIG\_L}: Configuration register (low part)}
\begin{tabular*}{\textwidth}{@{\extracolsep{\fill}} |c|c|c|c|c|c|c|c|}
 \hline
 \multicolumn{8}{|c|}{\texttt{CONFIG\_L} (0x00)}\\
 \hline
 BIT 7 & BIT 6 & BIT 5 & BIT 4 & BIT 3 & BIT 2 & BIT 1 & BIT 0 \\
 \hline
 \texttt{L\_EN} & \texttt{R\_EN} & \texttt{XTALK} & \texttt{AUTO\_INC} & \texttt{CONT\_MODE} & \texttt{CONV} & \texttt{CONV\_FINISHED} & \texttt{unused}\\
 \hline
\end{tabular*}

\paragraph{} Description of the content of the register:
\begin{description}
 \item[\texttt{L\_EN}] \qquad \textbf{Readable / Writeable / Initialize at 1}\\
       Set this bit to 1 to enable the left sensor.\\
       Set this bit to 0 to disable the left sensor.

 \item[\texttt{R\_EN}] \qquad \textbf{Readable / Writeable / Initialize at 1}\\
       Set this bit to 1 to enable the right sensor.\\
       Set this bit to 0 to disable the right sensor.

 \item[\texttt{XTALK}] \qquad \textbf{Readable / Writeable / Initialize at 0}\\
       Set this bit to 1 to enable crosstalk compensation on both sensors.\\
       Set this bit to 0 to disable crosstalk compensation on both sensors.\\
       More information on crosstalk can be found in the VL53L0X API manual.
       %%TODO add reference.

 \item[\texttt{AUTO\_INC}] \qquad \textbf{Readable / Writeable / Initialize at 0}\\
       Set this bit to 1 to enable \iic auto incrementation of the registers.\\
       Set this bit to 0 to disable \iic auto incrementation of the registers.\\
       Auto incrementation of the registers will automatically increment the internal register pointer after a read or a write. The pointer will cycle through the configuration registers if it was initially pointing to a configuration register (and go back to the first config register if it reached the last config register). The pointer will cycle through the data registers if it was initially pointing to a data register (and go back to the first data register if reached the last data register).

 \item[\texttt{CONT\_MODE}] \qquad \textbf{Readable / Writeable / Initialize at 0}\\
       Set this bit to 1 to enable continuous measurement mode.\\
       Set this bit to 0 to disable continuous measurement mode.\\
       Continuous measurement mode will start a new measurement as soon as the last measurement is over. It will also raise the interrupt after each measurement (depending on the interrupt setting) if the interrupt was reset.

 \item[\texttt{CONV}] \qquad \textbf{Readable / Writeable / Hardware clearable / Initialize at 0}\\
       Set this bit to 1 to start a measurement or start continuous measurement.\\
       Set this bit to 0 to stop continuous measurement. If CONT\_MODE is disabled the sensor will automatically clear this bit once the measurement is over.

 \item[\texttt{CONF\_FINISHED}] \qquad \textbf{Read-only / Hardware settable / Hardware clearable / Initialize at 0}\\
       Hardware set to 1 once the conversion is over. \\
       Hardware set to 0 after an \iic read of a data register.
\end{description}

\subsubsection{\texttt{CONFIG\_H}: Configuration register (high part)}
\begin{tabular*}{\textwidth}{@{\extracolsep{\fill}} |c|c|c|c|c|c|c|c|}
 \hline
 \multicolumn{8}{|c|}{\texttt{CONFIG\_H} (0x01)}\\
 \hline
 BIT 7 & BIT 6 & BIT 5 & BIT 4 & BIT 3 & BIT 2 & BIT 1 & BIT 0 \\
 \hline
 \multicolumn{2}{|c|}{\texttt{INT\_MODE}} & \multicolumn{6}{c|}{\texttt{DURATION}}\\
 \hline
\end{tabular*}

\paragraph{} Description of the content of the register:
\begin{description}
 \item[\texttt{INT\_MODE}] \qquad \textbf{Readable / Writeable / Initialize at 0b00}
       \begin{description}
        \item[00] No interrupts. Interrupts are disabled.
        \item[01] Full Interrupt. Generates an interrupt everytime a sensor gets a new measurement.
        \item[10] Half Interrupt. Generates an interrupt once all enabled sensors got a new measurement.\\
              \textbf{Note :} If only one sensor is enabled the behavior is equivalent to 0b01.
        \item[11] Unused.
       \end{description}

 \item[\texttt{DURATION}] \qquad \textbf{Readable / Writeable / Initialize at 0b000011}\\
       Controls the time budget allocated to each sensor for it's measurement. The final value is
       $$ \texttt{DURATION}*3 + 20\ [ms]$$
\end{description}

\subsubsection{\texttt{ADDRESS}: \iic Slave Address}
\begin{tabular*}{\textwidth}{@{\extracolsep{\fill}} |c|c|c|c|c|c|c|c|}
 \hline
 \multicolumn{8}{|c|}{\texttt{ADDRESS} (0x02)}\\
 \hline
 BIT 7 & BIT 6 & BIT 5 & BIT 4 & BIT 3 & BIT 2 & BIT 1 & BIT 0 \\
 \hline
 \multicolumn{8}{|c|}{\texttt{ADDRESS}}\\
 \hline
\end{tabular*}

\paragraph{} Description of the content of the register:
\begin{description}
 \item[\texttt{ADDRESS}] \qquad \textbf{Readable / Writeable / Initialize at 0x42}\\
       \iic Address of the device
\end{description}

\subsection{Data Registers}
All the data registers contains the distance sensed by the two VL53L0X chips. The data are formatted on one byte, unsigned integer representing the distance in centimeters.
All these sensors are read-only.

\begin{description}
 \item[\texttt{Right} (0x10)] Distance measured by the right sensor.
 \item[\texttt{Left} (0x11)] Distance measured by the left sensor.
 \item[\texttt{Min} (0x12)] Minimum distance measured by the sensors.
 \item[\texttt{Max} (0x13)] Maximal distance measured by the sensors.
 \item[\texttt{AVG} (0x14)] Average distance measured by the sensors.
\end{description}
