In RUN mode if the LED is not forced off (see section \ref{sec:BootModes}) the sensor might indicate an error code by blinking the status LED a few times. This should normally not happen during normal operation and is mainly used for debug purposes.
\newline
\noindent
\begin{tabularx}{\textwidth}{|c|X|}
 \hline
 \thead{Number of blinks} & \thead{Reason}                                                                        \\
 \hline
 1                        & Sensor related error during the main execution loop.                                  \\
 \hline
 2                        & Sensor related error while configuring the sensor. (RUN mode only)                    \\
 \hline
 3                        & Could not perform device initialization.                                              \\
 \hline
 4                        & Could not load device specific settings.                                              \\
 \hline
 5                        & \makecell[lc]{In SPAD calibration mode: could not perform SPAD calibration.           \\In reset mod: could not get SPAD calibration data.\\Otherwise: could load SPAD calibration data.}\\
 \hline
 6                        & Could not perform reference calibration.                                              \\
 \hline
 7                        & \makecell[lc]{In offset calibration mode: could not perform offset calibration.       \\In reset mod: could not get offset calibration data.\\Otherwise: could not load offset calibration data.} \\
 \hline
 8                        & \makecell[lc]{In crosstalk calibration mode: could not perform crosstalk calibration. \\In reset mod: could not get crosstalk calibration data.\\Otherwise: could not load crosstalk calibration data.} \\
 \hline
\end{tabularx}

